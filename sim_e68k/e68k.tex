
\documentclass[a4,11pt]{article}

%%% THIS DOCUMENT USES HYPERLATEX version 2.0
%   It can be converted to a normal LaTeX2e document by
%     checking the lines after the comment: %% HYPERLATEX

%% HYPERLATEX
\usepackage{hyperlatex}

%% HYPERLATEX
%%%% Hyperlatex preamble
\htmltitle{e68k Online Guide}
\htmllevel{html3.2}
\htmladdress{\copyright \xlink{Miguel
    Filgueiras}{index.html}, Universidade do Porto, 1992--1997}
\htmldepth{4}
\htmlautomenu{3}
\htmlpanel{1}

%%%%%%%%%%%

\title{A Portable Environment for Programming in MC68000
Assembly}

\author{Miguel Filgueiras\\
\\
{\small LIACC, Universidade do
Porto}\\
{\small R. do Campo Alegre 823, 4150 Porto, Portugal}\\
{\small {\tt email: mig@ncc.up.pt}}\\
\\
}


\date{February 92; revised January 1997}

%%%%%%%%%
\newcommand{\esek}{{\bf e68k}}

\newcommand{\comm}[1]{{\tt #1}}
%%%%%%%%%

\begin{document}
\maketitle

\section{Introduction}

The need for providing a uniform environment for students that
start programming in an assembly language and that may use
computers with different processors led to the design and
implementation of \esek, the system described in the
sequel.
The fact that in previous years the computers
available to students of our department had a Motorola MC68000
processor and the obvious benefits in using existing lecture notes on its
assembly language (\cite{Toma91}) explain the choice of language. 
Note that the
%% HYPERLATEX
\xlink{sources of \esek}{e68k.tgz}
are public and that they may well
be used as a skeleton for building a similar environment for another
(not too different) language.

\section{Using \esek}

The main idea behind \esek\    is to provide the user with a
programming environment similar to the ones that are traditional for
languages like BASIC. Within the environment the user may issue
commands to
assemble a program, list it, run it (in normal- or debug-mode), clear the
memory, and so on.

One difference to traditional interpreter systems
is that there is no direct input of the program text from the keyboard.
Instead a file containing the text must be specified and it may be altered
by calling an editor (without leaving \esek). This is the reason for
\esek\    asking at the very start for the name of the file that
is to be considered initially. The user is free to work with other files,
although at any time only one file can be open.

\subsection{Some Commands}

A brief description of the more frequently used commands will be now
given. Note that the set of \esek\    commands may be easily
altered. The help facility of \esek\    gives a means of
knowing exactly what commands are available in a particular
implementation.

\begin{description}
\item[Getting help] The \comm{help} command, which may be used whenever
a command
is expected, gives the user information on which commands are
available at the moment.
\item[Assembling] The \comm{assemble} command opens the
file whose name was last given and assembles the program in it. There
will be error messages whenever the file cannot be opened or there
are syntactic errors in the text. This command can only be used when
there is no assembled program in the memory.
\item[Listing] The \comm{list} command produces a listing of the
program that has been successfully assembled, giving an error if there
is none. For each line \esek\    shows the line number, the memory
address at which the data or code is located in hexadecimal,
and the actual line in the program text. An address will
be followed either by a {\em p} (for {\em program-segment}), or by a
{\em v} (for {\em values}, or {\em data-segment}) --- more details on
this below. 
\item[Running] There are two different modes for running a
program. The {\em normal\/} mode, in which the user has no control
on what the program is doing, is entered by using the \comm{run}
command. The {\em debug} mode, which is explained
below, gives the user the possibility of
following in detail the program execution. It is entered through the
use of \comm{debug}, and it has a distinct set of commands (that is
why the prompt is different in this mode). Whatever the mode is,
at the end of the execution \esek\    issues a message reporting why the
program stopped.
\item[Editing] The \comm{edit} command allows the
user to make changes in the current file by calling a text editor. Which
editor is used by default is
implementation dependent (for instance, in a UNIX system
the shell environment will be searched for a variable EDITOR, which if
defined will be taken to specify the editor).
The \comm{setedit} command
may be used to inform \esek\    of the name of the editor to be called.
Whenever possible \esek\    warns the user when listing or debugging
a program for a file that has been changed after the assembling of
the program.
\item[Using a different file] When a program that has been successfully
assembled is no longer needed the \comm{clear} command will
make \esek\    forget it. To specify that a different file is to be treated
in subsequent assembling, listing, or editing operations
the \comm{file} command is used. It is an error to issue this command
while there is an assembled program in memory.
\item[Quitting] The command \comm{quit} terminates the execution of
\esek\    and returns control to the operating system 
\end{description}

A typical sequence of commands for preparing and running a new program
will be: \comm{edit}, to write the program text into the current file,
\comm{assemble}, to assemble it, and \comm{run} to execute it if there
were no errors during the assembling (otherwise the errors should
be corrected by calling again the editor). When the results are not the
expected ones, running the program in debug-mode will
certainly be helpful in locating the problem.

\subsection{Debugging}

The \esek\    debugger is modelled after the \comm{dbx} symbolic debugger
normally supplied with UNIX systems, its capabilities being, of
course, only a small subset of those of \comm{dbx}. The debugger
accepts a set of commands that enables the user to execute the
program in memory in a controlled way and also to inspect and
alter the (simulated) MC68000 registers and memory contents.
The help facility may be used to ascertain what commands are
available --- note that these will change when the program has
been run to completion as some of them would not make sense at
that time.

When the debugger expects a command
it will print the program line containing the instruction
that is about to be executed (or, when the program has terminated,
that has just been executed) and a
special prompt asking for a command. The most important
debugger commands are explained in the following description.

\begin{description}
\item[Step-by-step execution] The \comm{step} command will make \esek\    to
execute the assembler instruction in the current line.
\item[Turning the debugger off] To run the rest of program in normal-mode,
without user intervention, the \comm{nodebug} command is used,
while \comm{abort} will cause an immediate exit from the debugger.
\item[Using breakpoints] The user may set and delete breakpoints at
program lines containing executable instructions. The commands
\comm{break} and \comm{delete} are provided for that, breakpoints
being referred to by line numbers. The command \comm{cont}
continues the program execution until either a breakpoint is reached
or the program exits.
\item[Accessing registers] The command \comm{regs} prints the 
registers contents (in hexadecimal), whereas
\comm{setreg}  alters the contents of a
given register. In the latter case,  the user can specify an attribute 
(\comm{.B}, \comm{.W}, or \comm{.L} --- the default) after the
register name. The
value to which the register will be set must be written according to the
syntax of the assembler for numeric constants.
\item[Accessing memory] The command \comm{mem}
prints the contents of a sequence of bytes in
memory, while \comm{setmem} alters the contents
of such a sequence. With the former the user is asked for
an address (in the data-segment) and the length in bytes of the
sequence to be displayed. Memory contents will be displayed
in hexadecimal and as characters, the character '?' meaning either
itself or any non-printable character.
With the latter command the user is also asked for an address
(also in the data-segment) and for a sequence of values to be
loaded into memory. Both the addresses and the values must be given
according to the syntax of the assembler for numeric constants.
\item[Listings] The \comm{list} command works as described before, while
\comm{line} prints the current line.
\item[Restarting] The \comm{restart} command forces the execution of
the program from the beginning.
\end{description}

\section{The Assembly Language}

The assembly language accepted by \esek\    is similar to the language
described in \cite{HarmL85} although the following restrictions and
differences are to be taken into account.

\begin{description}
\item[Names and expressions] Mnemonics, attributes and register names
must be in capital letters. Names for labels may have lower-case letters,
digits and '\_' (the underline character), but must not have periods
and must begin with a letter (in either upper- or lower-case). Labels
must begin at the first column of the line, and semi-colons are not
to be used after them. For the
time being, expressions are not allowed.
\item[Comments] A line starting with an asterisk, preceded or not by
blanks is taken to be a comment and discarded. Any character after
a semi-colon in a line will  also be discarded.
\item[Attributes] Whenever an instruction either has only one possible
attribute, or may have different attributes that can be determined by
the assembler, \esek\    treats the instruction as unsized and does not
allow the user to specify an attribute.
\item[Unimplemented instructions] A very few MC68000 instructions
were not implemented, in particular those that can only be used in
MC68000 supervisor mode. A suitable error message is issued when
such an instruction is found.
\item[Directives and pseudo-instructions] The following are provided:
\begin{itemize}
\item \comm{OPNF} opens a file and returns a file number that will be
used for accessing it. It has two arguments: an address register, which
must contain the address of a sequence of bytes terminated with a null-byte
that specifies the name (or path) of the file, and a data register, whose
byte 0 should be 0 for reading, 1 for writing, or 2 for appending (or
creating) the file. In the end, the byte 0 of the destination register
is set either to the number associated with the file, or to $-1$ whenever
an error occurred (wrong initial value in the data register, too many
files open, or the file could not be opened).
\item \comm{CLSF} closes the file whose number is given in byte 0 of
the data register used as its argument.
\item \comm{GETC} may be used either with a single argument or with two
arguments. The arguments must be data registers. With a single argument a
byte is read from the standard input (normally the keyboard) into the
byte 0 of the register. With two arguments, a byte is read from the
file whose number (see \comm{OPNF}) is given in byte 0 of the source register
into the byte 0 of the destination register. In this case and if the
file number is invalid the value $-1$ is written into the byte 0 of the
destination register.
\item \comm{PUTC} has either a single argument or two arguments, which
must be data register(s). With a single argument this directive
puts the contents of the byte 0 of the register into the standard output
(normally the screen). With two arguments, byte 0 of the source register
specifies the file number (see \comm{OPNF}) where byte 0 of the
destination register is to be put. In this case, byte 0 of the destination
register will be set to $-1$ when the given file number is invalid.
\item \comm{EXIT} has no arguments and causes the termination of the
program being executed.
\item \comm{END} is just used to mark the end of the program text. Any
line after it will be disregarded.
\item \comm{DS} (for {\em data structure}), may have the usual
attributes for byte, word (the default), and long, and has a single
argument which is a numeric
constant specifying the number of positions of the given length to be
allocated in the data-segment.
\item \comm{DC} (for {\em data constants}), may also have the usual
attributes as for \comm{DS}, and its argument is a constant list
consisting of constants separated by commas. The constants will be put
in successive positions of the given length in the data-segment. When
the attribute is \comm{.B} a sequence of characters enclosed by double
quotes is taken to be a character string, and the ASCII codes of the
characters in it are treated as the constants to be stored. \end{itemize}
\end{description}

Although not being the case with most computers having a MC68000
processor, \esek\    makes a separation between two memory segments:
the {\em program-segment} and the {\em data-segment}. This is due to
the fact that the internal representation of MC68000 instructions in \esek\   
is not at all their binary codes. It would therefore be a bad idea to
give the user access to memory locations whose contents are different
from the expected ones. When assembling a program \esek\   
automatically keeps track of what is to be put in each of both
segments, even when executable instructions are mixed with data
storage directives. As it is usual with this kind of architecture, the
assembly program cannot access the program-segment for storing or
getting data and cannot execute instruction codes
stored in the data-segment.

The usual big-endian behaviour of the MC68000 is implemented,
irrespective of the processor running \esek, if \esek\  is compiled
with the correct options (as described in the next section).

Another point that is worth mentioning is that messages caused by
syntactic errors were intentionally left not too detailed. This may help
the user in learning the syntax and in being prepared to use those many
hermitic, closed and horrible systems that still users have to use in
real life.

\section{Implementation}

The original implementation of \esek\    was done using the C language
in a Sun-4 running
UNIX. The dependency with respect to UNIX was deliberately kept low
to increase the portability of the environment, and amounts to some
routines for checking the modification date of a file, for defining
the default editor, and for calling an editor (by executing a shell
command of the form {\tt editor filename}).

The \esek\  sources should be compiled with the correct options for the
actual processor and operating system. In some Unix environments the
options will be set automatically. In any case the 
%% HYPERLATEX
\xlink{header file}{e68k.h.txt}
should be checked in what concerns the definitions of {\tt BIGENDIAN}, {\tt
LITTLEENDIAN} and {\tt UNIX}, {\tt DOS} identifiers.

The information on the syntax of the assembler instructions and
on the commands accepted by \esek\    was as much
as possible given, in the program header file,  as initial values to arrays.
In this way, additions or modifications may be easily done. Also,
comments were provided as an explanation on what each routine is
expected to do, and whenever some obscure solution was used
(fortunately, there are not many of them).

\section{Acknowledgements}

I would like to express my sincere thanks to Pe\-dro Bal\-ta\-zar
Vas\-con\-ce\-los, who gladly accepted to undertake the fastidious
task of checking that \esek\    was executing in a honest way the
MC68000 instructions. He also started the work on the configuration
for different operating systems and processors.

Nelma Moreira and Rog\'erio Reis helped a lot in correcting the bugs
that Ana Paula Tom\'as very effectively detected.

\begin{thebibliography}{X}

\bibitem[A. P. Tom\'as 1991]{Toma91}
Ana Paula Tom\'as, Notas e Exerc\'{\i}cios de Pro\-gra\-ma\-\c{c}\~ao
em MC68000. Fa\-cul\-da\-de de Ci\-\^en\-ci\-as,
Uni\-ver\-si\-da\-de do Por\-to, 1991.

\bibitem[Harman and Lawson 1985]{HarmL85}
Thomas L. Harman and B. Lawson, {\em The Motorola MC68000
Microprocessor Family}. Prentice-Hall, 1985.

\end{thebibliography}

\end{document}


